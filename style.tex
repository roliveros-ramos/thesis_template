% \usepackage[T1]{fontenc}
\usepackage[utf8]{inputenc}
\usepackage{amsmath}       % innecesarios si se usa amsart/amsbook % USO
% \usepackage{amsfonts}
\usepackage{amssymb} % USO
\usepackage{bold-extra}
% \usepackage{amsthm}
\usepackage{latexsym}
\usepackage{emptypage}
% \usepackage{graphicx} % USO
% \usepackage{wrapfig}
% \usepackage{rotating}
\usepackage[section]{placeins}
\usepackage[usenames, dvipsnames]{color} % USO
% \usepackage{colortbl}
% \usepackage{flafter}
% % \usepackage{subfigure} % USOD
\usepackage{fancyhdr}
\usepackage{pdfpages}
% \usepackage{natbib}
% \usepackage[margin=20pt, font=small, labelfont=bf, labelsep=period]{caption}
% \usepackage{url}             % espaciado adecuado de las URL largas en las referencias
%\usepackage{path}            % espaciado y guinado adecuado para las rutas de directorios
% \usepackage[final]{listings} % codigo fuente
% \usepackage{appendix}        % mejor control sobre los apendices
% \usepackage{amsmath,amsthm,amstext,amssymb,amsfonts}
\usepackage{textcomp}   % extra symbols (textopenbullet) % USO

% \usepackage{rotating}

% \usepackage{tabularx}
% \usepackage{booktabs}
% 
\usepackage{etoolbox, tocloft}
%\patchcmd{\tableofcontents}{\MakeUppercase\contentsname}{\contentsname}{}{}
\preto\section{%
  \ifnum\value{section}=0\addtocontents{toc}{\vskip10pt}\fi
}
\preto\figure{%
  \ifnum\value{figure}=0\addtocontents{lof}{{\bfseries Chapter \thechapter\vskip10pt}\par}\fi
}
\preto\table{%
  \ifnum\value{table}=0\addtocontents{lot}{{\bfseries Chapter \thechapter\vskip10pt}\par}\fi
}

\usepackage[pagebackref=true]{hyperref}
\hypersetup{colorlinks=true, linkcolor=black, citecolor=black, filecolor=black, urlcolor=black, linktoc=page,
	pdftitle={End-to-end modelling approach for an ecosystem approach to fisheries in the Humboldt Current Ecosystem}, 
	pdfauthor={Ricardo Oliveros--Ramos}, 
	pdfsubject={PhD Thesis},
pdfkeywords={Ecosystem modelling, Humboldt Current Ecosystem, OSMOSE}}

\usepackage{cite} % CREO QUE USO
\usepackage[round]{natbib} % USO
\usepackage{lscape} % USO
\usepackage{multirow} % USO
\usepackage{csquotes}
% \usepackage{subfig} % USO
\usepackage{caption}
\usepackage{subcaption}
% \renewcommand\thesubfigure{\roman{subfigure}}
\usepackage{array,graphicx}
\usepackage{booktabs}
\usepackage{threeparttable}
% \usepackage{biblatex}
% \usepackage[backend=biber,style=authoryear,sortcites,sorting=ynt]{biblatex}
% \addbibresource{referencesthese.bib}
\usepackage{appendix}
\usepackage{setspace}
\usepackage{titlesec}
\usepackage{lipsum}

\newcommand*\rot{\rotatebox{90}}


% \bibliographystyle{authordate_perso.bst}
%\bibliographystyle{ecology_rocio.bst}
% \bibliographystyle{te.bst}
%\graphicspath{{/data/rociocokriging/}{/data/mes_images/}}

%% Maquetacion ------------------------------------------------------------
   \parskip 0cm
   \oddsidemargin 2cm
   \evensidemargin 2cm
   \topmargin 1cm
   \textheight 23.5cm
   \textwidth 15cm
   \voffset -1.5cm
   \hoffset -2cm
   \headheight 1cm
   \headsep 1cm
   \topskip 0cm
   
% 
% \setlength{\oddsidemargin}{35mm}
% \setlength{\evensidemargin}{25mm}
% \setlength{\voffset}{-0.725in}
% \setlength{\hoffset}{-1in}
% \setlength{\textwidth}{150mm}
% \setlength{\topmargin}{4mm}
% \setlength{\headheight}{10mm}
% \setlength{\headsep}{15mm}
% \setlength{\topskip}{0mm}
% \setlength{\textheight}{243mm}

% \renewcommand{\baselinestretch}{1.2}           % interlinado un poquito m�s amplio
\setstretch{1.25}


% incluir numero de capitulo en las numeraciones
\numberwithin{section}{chapter}
\numberwithin{equation}{chapter}
\numberwithin{figure}{chapter}
\numberwithin{table}{chapter}



% definitions
\newcommand{\RR}{\mathbb{R}}
\def\xpar{\tilde{x}}
\def\xmean{x_k}
\def\smean{\sigma_k}
\def\xdyn{\textbf{x}_k}
\def\sdyn{\textbf{s}_k}
\def\smax{\textbf{s}^{(\max)}_k}
\def\smin{\textbf{s}^{(\min)}_k}
\def\xfinal{\textbf{x}}
\def\sfinal{\textbf{s}}
\def\ccov{c_{cov}}
\def\mucov{\mu_{cov}}
\def\step{\sigma_{size}}

% % comandos
% \newcommand{\ve}[1]{\boldsymbol{#1}}                       % vectores
% \newcommand{\mat}[1]{\boldsymbol{\mathsf{#1}}}          % matrices (roman)
% \newcommand{\esp}[1] {\mathbb{E}\left[#1\right]}	% esperanza
% \newcommand{\var}[1] {\mathbb{V}\left[#1\right]}	% varianza
% \newcommand{\cov}[1] {\mathbb{C}\left[#1\right]}	% funci�n de covarianza
% \newcommand{\pr}[1] {\mathbb{P}\left[#1\right]}		% Probabilidad
% \newcommand{\alert}[1]{\noindent\textcolor{BrickRed}{\textbf{(#1)\\}}}
% \newcommand{\simiid}{\buildrel\rm iid \over\sim}        % ~ iid
% \newcommand{\R}{\texttt{R}}
% \newcommand{\inla}{\texttt{INLA}}
% 
% \DeclareMathOperator{\logit}{logit}
\DeclareMathOperator*{\argmax}{arg\,max}
% 
% \newcommand{\af}{\alpha}
% \newcommand{\ta}{\theta}
% \newcommand{\ld}{\lambda}
% \newcommand{\M}{I\!\!M}
% \newcommand{\I}{\mathbb I}
% \newcommand{\1}{\mathbf{1}}
% \newcommand{\pp}{\Phi}
% \newcommand{\N}{\mathbb N}
% \newcommand{\x}{{\bf x}}
% \newcommand{\argmax}{\mbox{ argmax}}
% \renewcommand{\topfraction}{0.85}
% \renewcommand{\textfraction}{0.1}
% \newcommand{\dd}{\mathrm{d}}
% \newcommand{\T}{{\sf{T}}}
% \newcommand{\vecx}{\mathbf{x}}
% \newcommand{\vecy}{\mathbf{y}}
% \newtheorem{definition}{Definition}
% \newenvironment{example}{\noindent{\bf Example\/} \rm}{\hfill \\}
% \newenvironment{remark}{\noindent{\bf Remark\/} \rm}{\hfill \\}
% 
% \def\mm#1{\ensuremath{\boldsymbol{#1}}} % version: amsmath

%% Comandos necesarios para hacer el glosario
%\def\glossaryname{Glosario de s\'{\i}mbolos}
%\makeglossary
% 
% % This is done with the titlesec package                                                                            
%   \titleformat{\chapter}[hang]
% 	      {\thispagestyle{empty}}
%               {\normalfont\Large\filcenter} % {fmt}                                                                   
%               {\thechapter.\ }              % {label}                                                                 
%               {1pc}                         % {sep}                                                                   
% %               {\vspace{-1in}\enlargethispage{-0.5in}\thispagestyle{empty}}    % {before}                                  
% % 



\pagestyle{fancy}

 \fancyhead{}
        \fancyfoot{}
        \fancyhead[LO]{\rightmark}
        \fancyhead[RE]{\leftmark}
        \fancyhead[LE,RO]{\thepage}
        \renewcommand{\chaptermark}[1]{\markboth{\thechapter.\ #1}{}}
        \renewcommand{\sectionmark}[1]{\markright{\thesection\ #1}}

%
%\makeatletter
%\if@twoside
%  \renewcommand*{\chaptermark}[1]{%
%    \markboth{%
%      \ifnum\value{secnumdepth}>-1 %
%        \thechapter\ %
%      \fi
%      #1%
%    }{}%
%  }%
%\else
%  \renewcommand*{\chaptermark}[1]{%
%    \markright{%
%      \ifnum\value{secnumdepth}>-1 %
%        \if@mainmatter  
%          \thechapter\ %
%        \fi
%      \fi
%      #1%
%    }%
%  \fi
%\makeatother

% my new commandas

\newcommand{\includepaper}[1]{
\cleardoublepage
\setlength{\voffset}{0cm}
\setlength{\hoffset}{0cm}
\newpage
\includepdf[pages=-]{#1}
\setlength{\voffset}{-1.5cm}
\setlength{\hoffset}{-2cm}
\clearpage	
}

\newcommand{\includeappendix}[1]{
\clearpage
\setlength{\voffset}{0cm}
\setlength{\hoffset}{0cm}
\newpage
\includepdf[pages=-]{#1}
\setlength{\voffset}{-1.5cm}
\setlength{\hoffset}{-2cm}
\clearpage	
}