
\section*{Confronting ecosystem models to data}

Confronting ecosystem models to data is essential to increase their credibility and to start using them in support to management decision. A successful model calibration implies several computational, theoretical and practical challenges. The \texttt{calibrar} R package intends to provide a framework to simplify the calibration of complex models, in particular stochastic ones, for which fewer developments have been done compared to those for deterministic and differentiable models. There is “no free lunch” in optimization, and no optimization algorithm will perform bestfor every type of optimization problems (Wolpert and Macready 1997). In this direction, the next step we envisage is the inclusion of other optimization algorithms in our \texttt{calibrar} package so the users are provided with a suite of tools to solve a variety of calibration problems in a transparent way without additional technical complications. In parallel, more tests with other models and real-world calibration problems are required to improve the generality of the package, its flexibility and the robustness of the optimization algorithm. 


However, calibration is only one step in the process of rigorous model development and application. One important step we did not prioritize here due to time constraints, is a full sensitivity analysis for the OSMOSE model, which could have greatly helped in the calibration process by providing a rationale to reduce the number of parameters to estimate (Megrey et al. 2007, Dueri et al. 2012, Lehuta et al 2013). Complementary to this first approach and expected to be developed in close future is an uncertainty analysis on the parameters of the E2E model, relying on the calibration phase but also on the analysis of the uncertainties due to the different component models (ROMS-NPZD, OSMOSE, climate niches) and how they can combine together into an assessment of different management scenarios in the HCE. Additionally, a successful calibration does not mean that a model is reliable (Gaume et al. 1998), and a proper validation is always required, eventually providing information to improve the model and to revise the calibration (Walter and Pronzato 1997, Jorgensen and Bendoricchio 2001). In particular, a more detailed validation of our model results including alternative sources of information (e.g. trophic ecology) could help to increase the credibility of the model. 

\section*{Bridging ecosystem models with single species models}

Management strategy evaluation (MSE) is a set of simulation-based procedures to compare alternative management procedures (Butterworth 2007, De Lara and Martinet, 2008). More precisely, MSE consists in defining a set of operational objectives and to evaluate, by means of simulations, the performance of alternative management procedures in relation to the set of objectives defined, and taking into account the uncertainties related to the modeling process (Sainsbury et al. 2000, Butterworth 2007). One of the weakest points in some MSE applications is the use of the same model as the operative model (used to generate artificial data) and the assessment model (used to evaluate the status of the population given the artificial data generated by the assessment model). Even if the operative model is a more complex version of the assessment model, it is likely that the results would be artificially better than if an independent model with a different structure was used to generate the data. Additionally, by doing so, it is implicit that the assumptions of the model are correct, i.e. the reality is driven by the processes included in the operative model, which can lead to important model misspecifications and bias in the MSE because model uncertainty would not have been taken into account. In particular, when using single-species models as operative models, the impact of species interactions and their variability is not taken into account (or at least not directly) which could bias long term projections. In this respect, future MSE in the HCE could rely on a fully calibrated ecosystem model as the operative model while still using the current single-species assessment models for management purposes.  


A weakness of some single-species models is to take as constants some parameters which are expected to have a strong variability, like natural mortality, which greatly depends on the changes in the environmental conditions and trophic structure of the ecosystem, as well as the life stage of the individuals. Ecosystem models can provide information about such variability, which can be incorporated as a forcing or additional source of uncertainty in single-species models. By doing so, it would be possible to understand the impact of other sources of variability (e.g. the environment) in single-species models, which normally rely on the (phenomenological) estimation of time series of deviates to incorporate the impact of such sources of variability. On the other hand, keeping multi- and single-species models independent (not using outputs from one as input to the other) allows us to provide insights into the impact of different assumptions and resolution of the model processes and structure, e.g. single-species models being more fishery oriented (e.g selectivity modeling) and ecosystem models being more trophic oriented (e.g. predation modeling). 


\section*{Applications to EBFM}


Despite some attempts to move towards EBFM around the world, most commercial species remain managed using single-species management procedures (MP). Therefore, one interesting application would be to evaluate the impact of neglecting the interspecific interactions in the ecosystem as well as the interactions between single species MPs leading to concrete multispecies management strategies. This can be done by replicating the single-species management procedures using ecosystem models to eventually help to develop more robust MPs in an ecosystem context. Additionally, current single-species reference points (RP) cannot take into account important ecosystem processes, particularly here in the HCE the environmentally-induced changes in the ecosystem structure. Ecosystem models will allow to estimate ecosystem RPs taking into account the complex dynamics of the ecosystems while simultaneously allowing to work in a single-species context. However, by counting with ecosystem-based RPs (like multi-species MSY) and operational ecosystem models, a further step would be to carry out an integrated ecosystem MSE. Since several criteria are used to estimate "optimal" strategies for fisheries management, but most of them rely on single-species models, the solutions under the same criteria can be totally different by considering an ecosystem approach. Furthermore, ecosystem models can be better at forecasting the impacts of the environmental variability in the dynamics of exploited populations. Currently, physical models can provide synthetic scenarios of natural and human-induced environmental variability which can be used for MSE purposes. On the other hand, single-species models normally need to rely on simpler time-series of environmental variability to force the models, while spatial ecosystem models, like OSMOSE, can better exploit the variability given by ocean and biogeochemical models.


\section*{OSMOSE modelling platform}


The version of OSMOSE implemented in this thesis (OSMOSE 3 release 1, \url{www.osmose-model.org}) includes several improvements with respect to the versions used in previous published applications (OSMOSE 2), particularly related to the incorporation of interannual variability in the model. The OSMOSE model for the NHCE is also the first interannual application using OSMOSE. However, in all OSMOSE versions, the impact of fishing is simplified as it does not explicitly handle multiple fisheries but just species catches or fishing mortalities. OSMOSE 3 includes more flexible selectivity specifications (in addition to the original knife-edge selectivity), but a more general approach is needed to explore options for real management applications in a multiple fisheries context, e.g. one species being targeted by more than one fisheries (possibly in different areas) and one fishery targeting more than one species (possibly, with different selectivities and catchabilities). This approach will lead fisheries to be modeled similarly to other predator species which can have access to all the other species inside the limits specified by the size-specific predation hypothesis of OSMOSE.


Another improvement to bringing OSMOSE to better represent the NHCE ecosystem dynamics is a finer specification of land-based predators, like mammals and birds. This can be handled in OSMOSE 3 by using a time-varying size-specific natural mortality instead of a constant one as is the case in the current implementation. This approach will need to specify i) a proxy of the natural mortality induced by the land-based predators (e.g. time series of abundance or consumption), ii) the shape of the selectivity of the predator (equivalent to the ratios for the size-dependent predation for other predators), iii) the target preys (equivalent to the accessibility matrix for other predators) and iv) the estimation of the average natural mortality induced by the land-based predators during the calibration process. For example, data is available in the NHCE to apply such approach for seabirds preying upon anchovy. Other possibility for the inclusion of land-based predators is to assimilate their parameterization to that of a "fishery" in a multi-fisheries implementation as discussed before.


Finally, the interannual implementation of OSMOSE required to model the spatial distribution of fish, which is one of the current forcings in OSMOSE, and to construct interannual maps of fish distributions. However, fish spatial distribution is currently disrupted as discrete maps drive fish spatial dynamics. To reduce the potential disruptive effect, we refined the forcing by using seasonal maps for most of the species (four maps per year), but further details and mechanisms should be included in the movement sub-model in OSMOSE, particularly to smooth the transitions between maps and to not lose the spatial structure created in the model.


\section*{HCE modelling} 


A model cannot be better than the information used to build it. Ecosystem models in particular rely on a great quantity and quality of information in order to be able to properly reproduce the observed dynamics of the ecosystem and identify the parameters of the model. In this sense, gathering the information needed to build an interannual OSMOSE model for the Northern Humboldt Current Ecosystem was by itself a complex task, requiring a lot of pre-processing work in terms of data standardization. To review and standardize the information needed for this application has motivated the launch of an on-going IMARPE project ("Estimation of fishery-biological parameters for the sustainable management of marine resources", funded by RM-350-2013-PRODUCE) that has the objective to improve the IMARPE's database (IMARSIS) and to digitalize and concentrate all the sources of information pertinent for modeling purposes. Also, improvements in the standardization of abundance indices are needed, as the main source of information on the variability of the exploited populations. Currently, two master theses are under way at IMARPE to address i) a review of the fishery-independent abundance indices for Jack mackerel and ii) the development of empirical echo-abundance indices when length data from surveys is not appropriate to estimate biomass from acoustic data, mainly for less common non-commercial species.


In terms of the spatial distribution modelling, the current implementation of OSMOSE-NHCE uses the same distribution for all the schools of the same species, independently of the age or size. As data exist for different stages of anchovy (larvae, juveniles, recruits and adults), another master thesis is in progress to refine the spatial distribution models. Also, munida (squad lobster) requires more detailed modeling of its spatial distribution since the larger individuals start to develop demersal habits in comparison to the smaller ones which are pelagics off Peru. Finally, a more detailed pattern oriented validation is needed for all modelled species to ensure that the maps produced fulfill the requirements for our modeling objectives.


\section*{References}

De Lara M. and Doyen L. 2008. Sustainable Management of Natural Resources. Mathematical Models and Methods. Springer-Verlag, Berlin, 2008.

Dueri S., Faugeras  B., Maury O.,2012. Modelling the skipjack tuna dynamics in the Indian Ocean with APECOSM-E – Part 2: Parameter estimation and sensitivity analysis. Ecological Modelling 245:55-64.

Gaume E., Villeneuve J.-P., Desbordes M., 1998. Uncertainty assessment and analysis of the calibrated parameter values of an urban storm water quality model. Journal of Hydrology 210: 38–50.

Lehuta S., Petitgas P., Mahévas S., Huret M., Vermard Y., Uriarte A.,  Record N.R.,2013. Selection  and  validation  of  a  complex  fishery  model  using  an  uncertainty hierarchy. Fisheries Research  143:57–  66.

Jorgensen S.E. and Bendoricchio G., 2001. Fundamentals of Ecological Modelling. Third Edition. Elsevier. 530pp.

Megrey B.A., Rose K.A, Klumb R.A, Hay D.E, Werner F.E, Eslinger D.L, Smith S.L., 2007. A bioenergetics-based population dynamics model of Pacific herring (Clupeaharenguspallasi) coupled to a lower trophic level nutrient-–phyto\-plankton–-zoo\-plankton model: Description, calibration, and sensitivity analysis. Ecological Modelling 202:144–164.

Sainsbury K. J., Punt A. E. and Smith A. D. M. 2000. Design of operational management strategies for achieving fishery ecosystem ob¬jectives. ICES Journal of Marine Science, 57: 731-741.

Walter E. and Pronzato L., 1997. Identification of parametric models from Experimental data. Springer Masson. 413pp.

Wolpert, D.H. and Macready, W.G. (1997) No Free Lunch Theorems for Optimization. IEEE Transactions on Evolutionary Computation, 1(1):67-82.


