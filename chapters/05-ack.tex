I want to start by thanking my thesis' director, Yunne Shin. The first thing I cannot imagine now is to do my thesis with another person as my director. I am really thankful for all the long discussions and the support and comprehension during the last years. I think doing a PhD is not only an intellectual exercise but also a personal challenge, and Yunne was always there when I needed, thank you so much for that. También quiero agradecer a mi co-director de tesis, Arnaud Bertrand, por todo el apoyo durante los últimos años, incluso antes de que comenzara la tesis. No hubiera hecho el doctorado en Francia de no ser por la motivación de Arnaud, y aunque al final no haya pasado tanto tiempo allí como quisiera, su apoyo fue muy importante en los momentos críticos. Merci pour tout !

I also want to thank to the reviewers of my thesis and jury: Stephanie Mahevas, Jean-Christophe Poggiale, Ray Hilborn, Philippe Cury, Jorge Csirke and Coleen Moloney; for their valuable comments and interesting discussion. The defense of my thesis has been one of the moments I have enjoyed the most of my life, many thanks for that.

I thank to IRD for the funding during my thesis, which make it possible to work in three different countries (South Africa, France and Peru) and to attend to several conferences around the world. This has been a great experience and a huge oportunity to learn. I particularly thank the support of Philippe \mbox{Verley} (IRD) to my research, which was unvaluable to complete my thesis objectives, many thanks for all the time invested. I also thank to Vincent Echevin (IRD), David Kaplan (IRD) and David Mouillot (UM2) for their help during my thesis commitees and my research. I also thank the funding from LMIs \mbox{DISCOH} (Perú) and \mbox{ICEMASA} (South Africa) and the EMIBIOS project. I have had the oportunity to attend several workshops and summer schools during my thesis which were very valuable for the development of my work. I greatly appreciate the finantial support of the NIMBioS (National Institute for Mathematical and Biological Synthesis, USA) for the participation on the High Performance Computing tutorial, to the MBI (Mathematical Biosciences Institute, USA) for the funding to participate in the summer school ``Stochastics Applied to Biological Systems'' and to the PICES (North Pacific Marine Science Organization) for the participation in the summer school ``End-to-End Models for Marine Resources Management and Research''. I'll make all the money spent on my scientific training worth.

Quiero agradecer al Instituto del Mar del Perú (IMARPE), como ente abstracto, por haberse constituido como el espacio en donde me formé como investigador. El IMARPE me ha brindado muchas facilidades para realizar esta tesis, pero también muchas dificultades, y ambas --especialmente las últimas-- han sido decisivas para mi trabajo y mi formación como científico. Mi visión de lo que significa hacer Ciencia desde un país en desarrollo no sería la misma si no hubiera estado involucrado en el IMARPE como lo he estado durante los últimos años. Quiero agradecer en especial a Jorge Tam por haberme motivado a quedarme y guiarme en mis inicios en el Instituto. A Renato Guevara-Carrasco por su apoyo y motivación, muchas gracias por todos los consejos y ayudarme a no perder de vista que la Ciencia también tiene un lado humano. A todos mis amigos CIMOBPers (Dante, Yvan, Jorge, Carlos, Criscely y Wencheng) por compartir el inicio de la aventura; y mis amigos de Dinámica (Erich, Enrique, Pablo, Giancarlo, Crisi, Wenchito y Josymar) por compartir el final, que siempre es más difícil. 

I essentially did all my thesis in the amazing Mother City: Cape Town. And the second thing I just cannot imagine is working on my PhD at any other place than the University of Cape Town. I have to double thank Coleen Moloney for opening to me the doors of her lab, and to Gilly Smith for facilitating my stays. Thank you for welcoming me at UCT. To all the people I met at Moloney's lab (Dino, Shannon, Margit, Hilkka, Grea, Louis, Nandipha, Saachi) for sharing the mutual experience, specially to Dino for being such a nice office partner and forcing me to swim. Next door, to Astrid and Rachel for helping to my work. I have to specially thank to J\'{e}r\^{o}me, for forcing me to wake up in a normal scheadule (aka ALL the lifts), for forcing me to go out more often than I'd normally would, for sharing the office when I don't have keys but, more broadly, for sharing the experience of being a foreign \emph{doctorant} in Cape Town (merci beaucoup). To Philippe and Ainhoa, for being there and caring, I didn't know what language to write this part, I really think should be Spanish but the feeling is the same: muchas gracias por todo, amigos! To Ad\'ela\"{\i}de (the original tango partner) and Laure for being such a good friends (and for ALL the admin help, Laure, thanks so much!). I'm pretty sure most of my ``sciencing'' was done at my home in Cape Town: the Blake House. And that wouldn't be possible without the housemates. This would be such a long list, but I want to specially thanks to Caitlin for make it feel home, for real; to Mirette (the original sushi and cooking partner) and to that-french-girl Paris (aka Alexia). To everybody, thank you for making this possible, for your friendship and for making my south african experience unforgettable. Ubuntu.

I did not spend as much time in France as it was scheaduled, or as I would have liked. However, my time there was very important for the process. Being at S\`ete was probably the only moment I really feel as a student, I would have missed something important if not there. I met such a nice people there, but I particularly appreciate the long talks (in english), thanks Alexandra and Mariana for that (I'll learn french at some point). Por las largas charlas en español, gracias a Ana, Giannina y Rocío. Y por limpiar los desastres mientras cocinaba (Ana y Giannina) o obligarme a limpiarlos por mi mismo (Rocío). A Rocío especialmente por soportarme como \emph{housemate}.

Lastly but not least important, finishing is the most difficult part of a lot of things, and a doctoral thesis is not the exception. After such a long time, sometimes we need external forces to help to give the last push. I want to specially thank to Claire Antel, for providing me an additional source of stress, big enough to help me finishing my thesis last minute but on time ('cause stress is good and you always need a bit more to make things happen). A very limited set of choices could have made a better end of thesis; muchas gracias, Clarita. En la misma línea, quiero agradecer a Patricia Alcántara Pizarro, por ayudarme a encontrar la motivación para hacer algo mínimamente decente para mi defensa (en todos los sentidos posibles). Aunque fuera a último momento, fue más de lo que necesitaba. Cada persona cruza en nuestra vida por algo, con suerte más de una vez. Muchas gracias por eso.

Finalmente, quiero expresar mi profundo agradecimiento a mi familia, por haberme apoyado siempre y alentarme a seguir la vida académica. Agradezco a mis abuelos, Heraclio y Julia, por enseñarme desde pequeño a valorar el estudio y el conocimiento; y a mis padres, Ricardo y María Teresa, por brindarme siempre un ambiente de motivación intelectual y apoyarme sin objeciones en todas los objetivos que he emprendido. A mi querida hermana Andrea y a mi tía Ana María, por apoyarme siempre que lo necesito y hacer mi vida más sencilla. 

Para todos ellos, mi más sincera gratitud.

