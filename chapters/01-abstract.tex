
This work represents an original contribution to the methodology for ecosystem models’ development as well as the first attempt of an end-to-end (E2E) model of the Northern Humboldt Current Ecosystem (NHCE). The main purpose of the developed model is to build a tool for ecosystem-based management and decision making, reason why the credibility of the model is essential, and this can be assessed through confrontation to data. Additionally, the NHCE exhibits a high climatic and oceanographic variability at several scales, the major source of interannual variability being the interruption of the upwelling seasonality by the El Niño Southern Oscillation, which has direct effects on larval survival and fish recruitment success. Fishing activity can also be highly variable, depending on the abundance and accessibility of the main fishery resources. This context brings the two main methodological questions addressed in this thesis, through the development of an end-to-end model coupling the high trophic level model OSMOSE to the hydrodynamics and biogeochemical model ROMS-PISCES: i) how to calibrate ecosystem models using time series data and ii) how to incorporate the impact of the interannual variability of the environment and fishing.


First, this thesis highlights some issues related to the confrontation of complex ecosystem models to data and proposes a methodology for a sequential multi-phases calibration of ecosystem models. We propose two criteria to classify the parameters of a model: the model dependency and the time variability of the parameters. Then, these criteria along with the availability of approximate initial estimates are used as decision rules to determine which parameters need to be estimated, and their precedence order in the sequential calibration process. Additionally, a new Evolutionary Algorithm designed for the calibration of stochastic models (e.g Individual Based Model) and optimized for maximum likelihood estimation has been developed and applied to the calibration of the OSMOSE model to time series data.


The environmental variability is explicit in the model: the ROMS-PISCES model forces the OSMOSE model and drives potential bottom-up effects up the foodweb through plankton and fish trophic interactions, as well as through changes in the spatial distribution of fish. The latter effect was taken into account using presence/absence species distribution models which are traditionally assessed through a confusion matrix and the statistical metrics associated to it. However, when considering the prediction of the habitat against time, the variability in the spatial distribution of the habitat can be summarized and validated using the emerging patterns from the shape of the spatial distributions. We modeled the potential habitat of the main species of the Humboldt Current Ecosystem using several sources of information (fisheries, scientific surveys and satellite monitoring of vessels) jointly with environmental data from remote sensing and in situ observations, from 1992 to 2008. The potential habitat was predicted over the study period with monthly resolution, and the model was validated using quantitative and qualitative information of the system using a pattern oriented approach.


The final ROMS-PISCES-OSMOSE E2E ecosystem model for the NHCE was calibrated using our evolutionary algorithm and a likelihood approach to fit monthly time series data of landings, abundance indices and catch at length distributions from 1992 to 2008. To conclude, some potential applications of the model for fishery management are presented and their limitations and perspectives discussed.
